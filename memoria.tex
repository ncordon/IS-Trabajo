\documentclass[a4paper,10pt]{article}
% Símbolo del euro
\usepackage[gen]{eurosym}
% Codificación
\usepackage[utf8]{inputenc}
% Idioma
\usepackage[spanish]{babel} % English language/hyphenation
\selectlanguage{spanish}
% Hay que pelearse con babel-spanish para el alineamiento del punto decimal
\decimalpoint
\usepackage{dcolumn}
\newcolumntype{d}[1]{D{.}{\esperiod}{#1}}
\makeatletter
\addto\shorthandsspanish{\let\esperiod\es@period@code}
\makeatother
% To work with bibtex
%\usepackage{natbib}
\usepackage[backend=bibtex,style=numeric,sorting=none]{biblatex}
\bibliography{references}
\usepackage{longtable}
\usepackage{tabu}
\usepackage{supertabular}

\usepackage{multicol}
\newsavebox\ltmcbox

% Para algoritmos
%\usepackage{algorithm}
%\usepackage{algorithmic}
\usepackage{amsthm}
% Para matrices
\usepackage{amsmath}

% Símbolos matemáticos
\usepackage{amssymb}
\let\oldemptyset\emptyset
\let\emptyset\varnothing

% Hipervínculos
\usepackage{url}

\usepackage[section]{placeins} % Para gráficas en su sección.
\usepackage{mathpazo} % Use the Palatino font
\usepackage[T1]{fontenc} % Required for accented characters
\newenvironment{allintypewriter}{\ttfamily}{\par}
\setlength{\parindent}{0pt}
\parskip=8pt
\linespread{1.05} % Change line spacing here, Palatino benefits from a slight increase by default


% Imágenes
\usepackage{graphicx}
\usepackage{float}
\usepackage{caption}
\usepackage{wrapfig} % Allows in-line images

% Referencias
\usepackage{fncylab}
\labelformat{figure}{\textit{\figurename\space #1}}

\usepackage{hyperref}
\hypersetup{
  colorlinks   = true, %Colours links instead of ugly boxes
  urlcolor     = blue, %Colour for external hyperlinks
  linkcolor    = blue, %Colour of internal links
  citecolor   = red %Colour of citations
}

%Basado en: http://en.wikibooks.org/wiki/LaTeX/Theorems
\usepackage{amsthm}
\newtheorem*{mydef}{Definición}
\newtheorem{mydefn}{Definición}
\newtheorem{theorem}{Teorema}
\everymath{\displaystyle} % Displaystyle por defecto

% To include code
\usepackage{xcolor}
\usepackage{listings}

% code in bash style
\lstdefinestyle{BashInputStyle}{
  language=bash,
  basicstyle=\small\ttfamily,
  numberstyle=\tiny,
  numbersep=3pt,
  columns=fullflexible,
  backgroundcolor=\color{gray!20},
  xleftmargin=0.1\linewidth
}


% To change level of indentation
\newenvironment{answer}{%
\begin{list}{}{%
}%
\item[]}{\end{list}}


\makeatletter
\renewcommand{\@listI}{\itemsep=0pt} % Reduce the space between items in the itemize and enumerate environments and the bibliography
\newcommand{\imagent}[4]{
  \begin{wrapfigure}{#4}{0.7\textwidth}
    \begin{center}
    \includegraphics[width=0.7\textwidth]{#1}
    \end{center}
    \caption{#3}
    \label{#4}
  \end{wrapfigure}
}

\newcommand{\imagen}[4]{
  \begin{minipage}{\linewidth}
    \centering
    \includegraphics[width=#4\textwidth]{#1}
    \captionof{figure}{#2}
    \label{#3}
  \end{minipage} 
}

%Customize enumerate tag
\usepackage{enumitem}
%Sections don't get numbered
%\setcounter{secnumdepth}{0}


\usepackage{fontspec}
\setmainfont{Arial}
\usepackage{geometry}
 \geometry{
 a4paper,
 total={210mm,297mm},
 left=30mm,
 right=30mm,
 top=25mm,
 bottom=25mm,
 }
 
 
% To include code
\usepackage{xcolor}
\usepackage{listings}

% code in bash style
\lstdefinestyle{BashInputStyle}{
  language=bash,
  basicstyle=\small\ttfamily,
  numberstyle=\tiny,
  numbersep=3pt,
  columns=fullflexible,
  backgroundcolor=\color{gray!20},
  xleftmargin=0.1\linewidth
}
\renewcommand*{\newunitpunct}{\newline\space}
\begin{document}
%%%%%%%%%%%%%%%%%%%%%%%%%%%%%%%%%%%%%%%%%
% University Assignment Title Page 
% LaTeX Template
% Version 1.0 (27/12/12)
%
% This template has been downloaded from:
% http://www.LaTeXTemplates.com
%
% Original author:
% WikiBooks (http://en.wikibooks.org/wiki/LaTeX/Title_Creation)
% Modified by: NCordon (https://github.com/NCordon)
%
% License:
% CC BY-NC-SA 3.0 (http://creativecommons.org/licenses/by-nc-sa/3.0/)
% 
% Instructions for using this template:
% This title page is capable of being compiled as is. This is not useful for 
% including it in another document. To do this, you have two options: 
%
% 1) Copy/paste everything between \begin{document} and \end{document} 
% starting at \begin{titlepage} and paste this into another LaTeX file where you 
% want your title page.
% OR
% 2) Remove everything outside the \begin{titlepage} and \end{titlepage} and 
% move this file to the same directory as the LaTeX file you wish to add it to. 
% Then add \input{./title_page_1.tex} to your LaTeX file where you want your
% title page.
%
%%%%%%%%%%%%%%%%%%%%%%%%%%%%%%%%%%%%%%%%%
\begin{titlepage}

\newcommand{\HRule}{\rule{\linewidth}{0.5mm}} % Defines a new command for the horizontal lines, change thickness here

\center % Center everything on the page
~\\~\\~\\~\\~\\
\bigskip
%----------------------------------------------------------------------------------------
%	HEADING SECTIONS
%----------------------------------------------------------------------------------------
\textsc{\LARGE Universidad de Granada}\\[1.5cm]
\textsc{\Large Ingeniería de Servidores}\\[0.5cm] 

%----------------------------------------------------------------------------------------
%	TITLE SECTION
%----------------------------------------------------------------------------------------
\bigskip
\HRule \\[0.4cm]
{ \huge \bfseries IOStat,\\ comprobación experimental de su funcionamiento y significado de algunas medidas
que más afecten al rendimiento	}\\[0.4cm] % Title of your document
\HRule \\[1.5cm]
 
%----------------------------------------------------------------------------------------
%	AUTHOR SECTION
%----------------------------------------------------------------------------------------

\begin{minipage}{0.4\textwidth}
\begin{center} \large
%\emph{Ignacio Cordón Castillo}\\
\end{center}
\end{minipage}

%----------------------------------------------------------------------------------------
%	LOGO SECTION
%----------------------------------------------------------------------------------------

\begin{center}
\includegraphics[width=9cm]{./ugr.jpg}
\end{center}
%----------------------------------------------------------------------------------------

\vspace{\fill}% Fill the rest of the page with whitespace
\large\today
\end{titlepage}  

\newpage
\thispagestyle{empty}
\tableofcontents
\newpage
\setcounter{page}{1}
% Examples of inclussion of images
%\imagent{ugr.jpg}{Logo de prueba}{ugr}
%\imagen{ugr.jpg}{Logo de prueba}{ugr2}{size relative to the \textwidth}


\section{Resumen}
Monitorizar el empleo de discos es una función básica en las competencias de un administrador de un servidor, o incluso
podría ser aprovechado por un usuario medio.

A lo largo de este texto se analizará un monitor de dispositivos llamado \texttt{IOStat}, que viene por defecto instalado
en algunos sistemas UNIX y se constatará mediante la ejecución de un experimento en que se medirán distintos tipos de medios 
de almacenamiento mediante un experimento confeccionado a tales fines: se probarán distintos tipos de disco
duro: IDE,SATA; distintos formatos de partición: NTFS, FAT32,\ldots; y distintas revoluciones por minuto: 3600 rpm, 5400 rpm y 7200 rpm.

Asimismo, también efectuaremos un experimento en el que analizaremos su uso como monitor de CPUs: se probarán varios
tipos de CPU, variando factores como la velocidad(en Hz), el número de cores, \ldots para correr un programa del que es
conocido que somete a la CPU a mucha carga.
\section{Introducción}
Cuando como administradores de un sistema estamos en la obligación de ofrecer una determinada calidad de servicio (cubrir
necesidades mínimas de los usuarios que emplearán el sistema), y deseamos mejorar tanto la utilización de CPU como la utilización de 
memoria de disco, podemos usar un monitor de CPU o de dispositivos, respectivamente.

La pregunta es por qué es deseable optimizar esta parte de un sistema operativo, cuando hay muchos más campos que se pueden
mejorar en un sistema. La respuesta a esta pregunta se basa en que en el empleo de disco se encuentran los mayores cuellos de
botella en un ordenador. Porque una transferencia a  disco es del orden de 2000000 de veces más lenta que una transferencia a
caché de nivel L1, y del orden de 40 veces más lenta que la memoria RAM, y por tanto es potenciablemente optimizable. \cite{conferencia}

IOstat es un ejemplo de estos monitores. Se trata de un sampling monitor (monitor que funciona a intervalos regulares de
tiempo), de tipo software, que recoge información sobre el empleo de CPU y de dispositivos, basándose en el empleo de la información almacenada
en los sistemas UNIX en \texttt{/proc}.

Para proporcionar una calidad de servicio determinada, en ocasiones es importante detectar los cuellos de botella en el empleo
de la memoria o de la CPU. Y \texttt{IOStat} puede ayudarnos a llevar a cabo ese fin. Por ejemplo, nos podría ayudar a rebalancear
la carga entre discos duros; si por ejemplo tenemos varios discos duros y uno de ellos se emplea más que el resto, en términos
de cantidad de datos totales escritos o leídos, puede ayudarnos a redistribuir los archivos a los que más se accede, localizando
el disco duro en el que se encuentran; o si incluso uno de los discos duros es más rápido que el resto (información que 
también puede estraerse de los informes que genera el monitor), podemos colocar los archivos de los usuarios que más uso generen en dicho
disco duro, o incluso los datos a los que más se accede, para optimizar el sistema. En IOStat no existe interacción con el 
administrador/analista, obteniendo sus datos de \texttt{/proc/diskstats}
%(?).
En muchas ocasiones, los archivos más accedidos en un sistema son los del sistema operativo, y por tanto el monitor nos podrá
ayudar a determinar a qué disco o partición mover los archivos en función de la rapidez de cada una. \cite{merelo}

Su uso principal es la monitarización de la memoria, ya que para la monitarización de la CPU existen herramientas mucho más
completas, como el comando de UNIX \texttt{top}, pero constituyen sólo  un front-end al sistema de archivos \texttt{/proc}, y no
puede mostrar información que un sistema UNIX de por sí tampoco pueda. \cite{readme}

\section{Memoria}
\subsection{IOStat \cite{maniostat}}
El comando \texttt{iostat}, del paquete \texttt{sysstat} está disponible en UNIX empleado para monitorizar la carga de E/S de un sistema,
así como el uso de la CPU. Para la monitorización E/S se basa en computar el tiempo que un dispositivo de E/S está activo en 
relación a su ratio medio de transferencia. El comando está escrito en C, y está disponible bajo licencia GNU Public License v2.

El comando genera dos informes estadísticos: uno sobre el uso de CPU y otro sobre uso de E/S. Para mostrar cada uno, ocultando
el otro, basta usar \texttt{iostat -c} o \texttt{iostat -d} respectivamente. Respecto a las estadísticas CPU, cabe mencionar
que si se está ejecutando el comando en un sistema multiprocesador, las estadísticas mostradas constituyen una media de todos
los núcleos. Por defecto se muestran ambos informes. La primera vez que se ejecuta el comando, muestra las estadísticas 
recopiladas desde la última vez que se reinició el sistema. Las sucesivas veces se muestran estadísticas acumuladas desde la
última vez que el comando reunió información sobre el sistema. La sintaxis del comando es:

  \begin{lstlisting}[style=BashInputStyle]
  iostat [opciones] [dispositivo] interval count
  \end{lstlisting}
  
  Donde \texttt{opciones} son los distintos flags que se le pueden pasar al comando, entre los que destacan, aparte de los ya
  mencionados:
  
  \begin{itemize}
   \item \texttt{-j \{ID | LABEL | PATH | UUID |\ldots\}}: muestra para cada dispositivo la etiqueta, el UUID,\ldots correspondiente,
   en función de lo que hayamos indicado en el comando en lugar del descriptor de bloques /dev/sdX-n usado para identificar a los dispositivos
   por defecto.
   \item \texttt{-k}: muestra las estadísticas expresándolas en kilobytes por segundo.
   \item \texttt{-m}: muestra las estadísticas expresándolas en megabytes por segundo.
   \item \texttt{-p}: proporciona estadísticas no sólo para cada dispositivo, sino también para las particiones presentes en cada
   dispositivo.
   \item \texttt{-t}: hace que en los informes se incluya también un timestamp con la hora a la que corresponden.
   \item \texttt{-x}: muestra estadísticas expandidas. Si no se usa esta opción, se muestran para cada dispositivo solamente
   los valores: \texttt{tps, kB\_read/s, kB\_wrtn/s, KB\_read, kB\_wrtn}. Si se emplea dicha opción se muestra toda la
   información descrita en \ref{optdisp}, excepto los campos mencionados anteriormente.
   \item \texttt{-y}: omite el primero de los informes (información acumulada desde el último reinicio).
   \item \texttt{-z}: omite en los informes aquellos dispositivos para los que no se registró actividad durante la acumulación
   de información.
  \end{itemize}

\texttt{interval} y \texttt{count} son dos números naturales. Son omitibles ambos, o sólo \texttt{count}. \texttt{interval} indica
a \texttt{iostat} la duración del intervalo de tiempo, en segundos, durante el cual debe recopilar información; una vez vencido
ese tiempo se mostrará por pantalla un reporte de información concerniente a ese periodo de tiempo. \texttt{count} indica al comando
cuántos informes se quieren. Por ejemplo, si se llama al comando de la forma:

\begin{lstlisting}[style=BashInputStyle]
  iostat 5 10
\end{lstlisting}
  
entonces \texttt{iostat} mostrará 10 informes, uno cada 5 segundos, tanto de estadísticas CPU como de estadísticas de memoria.

Si se omite el parámetro \texttt{count} entonces \texttt{iostat} mostrará información cada vez que venza el tiempo indicado en \texttt{interval},
hasta que se interrumpa la ejecución del comando.

Si se omiten ambos parámetros, se mostrará un único informe (CPU+memoria si no se ha indicado que se muestre uno de ellos solamente).

Por omisión se genera información para todos los dispositivos de memoria disponibles en el sistema, si no se indica lo contrario
con el parámetro \texttt{dispositivo}. Por ejemlo, la siguiente llamada a \texttt{iostat}:

\begin{lstlisting}[style=BashInputStyle]
  iostat /dev/sda /dev/sdb
\end{lstlisting}

generaría información únicamente para los dispositivos correspondientes a los descriptores \texttt{/dev/sda} y \texttt{/dev/sdb}.

\subsubsection{Informe CPU}
La información aportada incluye:
\begin{itemize}
 \item \texttt{\%user}: porcentaje de uso CPU en el nivel usuario (generados por la ejecución de aplicaciones).
 \item \texttt{\%nice}: porcentaje de uso CPU correspondiente a procesos con prioridad cambiada.
 % Expandir
 \item \texttt{\%system}: porcentaje de uso CPU en el nivel kernel.
 \item \texttt{\%iowait}: porcentaje de tiempo que la CPU ha estado ociosa esperando a peticiones E/S.
 \item \texttt{\%steal}: porcentaje de tiempo empleado en espera por un núcleo virtual mientras el hipervisor servía a otro
 núcleo virtual. Este parámetro es útil cuando se están realizando virtualizaciones, por ejemplo con Virtualbox, y el número de
 virtualizaciones simultáneas es mayor que el número de núcleos físicos en la máquina.
 \item \texttt{\%idle}: porcentaje de tiempo que la CPU estuvo ociosa mientras el sistema no tenía una petición E/S pendiente.
\end{itemize}

\subsubsection{Informe dispositivos}
\label{optdisp}
La información mostrada puede incluir (en función de si se emplea la opción \texttt{-x} o no):
\begin{itemize}
 \item{\texttt{Device}}: descriptor de dispositivo \texttt{/dev/sdX} o de partición \texttt{/dev/sdXn}, donde \texttt{X} es una
 letra única que identifica a cada dispositivo y \texttt{n} un número que identifica de forma unívoca cada partición para un
 dispositivo dado.
 \item{\texttt{tps}}: número de transferencias pro segundo a un dispositivo. Una transferencia es una petición E/S que puede
 incluir varias peticiones lógicas de E/S combinadas. Esto es, si se intentan leer varios datos desde disco duro simultáneamente,
 puede que se hallen en el mismo bloque de disco duro o en bloques contiguos, y varias peticiones de datos dan lugar a una única
 petición de E/S. Así, una petición de E/S no tiene un tamaño fijo.
 \item{\texttt{KB\_read/s}}: número de KB leídos del dispositivo por segundo.
 \item{\texttt{KB\_wrtn/s}}: número de KB escritos al dispositivo por segundo.
 \item{\texttt{KB\_read}}: número de KB leídos en total desde el dispositivo.
 \item{\texttt{KB\_wrtn}}: número de KB escritos en total al dispositivo.
 \item{\texttt{rrqm/s}}: número de peticiones de lectura combinadas por segundo.
 \item{\texttt{wrqm/s}}: núero de peticiones de escritura combinadas por segundo.
 \item{\texttt{r/s}}: número de peticiones de lectura(ya combinadas) completadas por segundo.
 \item{\texttt{w/s}}: número de peticiones de escritura(ya combinadas) completadas por segundo.
 \item{\texttt{rKB/s}}: equivalente a \texttt{KB\_read/s}.
 \item{\texttt{wKB/s}}: equivalente a \texttt{KB\_wrtn/s}.
 \item{\texttt{avgrq-sz}}: número medio de peticiones de lectura, en bloques (en los kernels más modernos 1 bloque equivale a 512KB) emitidas
 al dispositivo.
 \item{\texttt{avgu-sz}}: longitud media de la cola para las peticiones emitidas al dispositivo.
 \item{\texttt{await}}: media en milisegundos que tarda una petición en ser servida (incluyendo tiempo de servicio y de cola).
 \item{\texttt{r\_await}}: media en milisegundos que tarda una petición de lectura en ser servida.
 \item{\texttt{w\_await}}: media en milisegundos que tarda una petición de escritura en ser servida.
 \item{\texttt{svctm}}: media en milisegundos de tiempo de servicio para las peticiones E/S emitidas al dispositivo.
 \item{\texttt{\%util}}: bandwith del dispositivo.
\end{itemize}

\section{Evaluación de prestaciones con IOStat}
Vamos a efectuar un análisis de prestaciones sobre varios tipos de discos duros, determinando la velocidad de cada 
disco en el copiado de archivos. Los discos duros podrán tener particiones de diferente tipo y tamaño.

\subsection{Definición del sistema}
El principal objetivo del estudio es comparar la velocidad del copiado de archivos en varios discos duros con particiones
en distintos formatos. En otras palabras, determinar cuál es el tipo de almacenamiento más rápido. El estudio se centrará
por tanto en discos duros, conectados todos ellos (no simultáneamente) a un equipo con las siguientes características:

%Completar.

Obviaremos del equipo todo aquellos componentes que no interfieran en la transferencia de archivos. Nuestro sistema estará
conformado por tanto por varios discos duros interconectados por un bus de E/S.

\subsection{Servicios proporcionados por el sistema}
El servicio principal proporcionado por el sistema es la transferencia y lectura de datos.

\subsection{Métricas}
Dado que se dispone de limitaciones en los recursos hardware para evaluar las prestaciones, hay ciertos modelos de almacenamiento
que no serán estudiados. Para cada modelo de almacenamiento, se evaluará los recursos consumidos por la transferencia, así
como el tiempo invertido. Por tanto aparecen las siguientes métricas para evaluar las prestaciones:

\begin{itemize}
 \item Número de Kilobytes escritos en el dispositivo por segundo (\texttt{KB\_wrtn/s} en \texttt{IOStat})
 \item Número de peticiones de escritura completadas en el dispositivo por segundo.
 \item Tamaño de las peticiones de escritura al dispositivo.
 \item Número total de peticiones de escritura efectuadas al dispositivo.
 \item Tamaño de la cola en el dispositivo durante la transferencia.
 \item Tiempo de espera de las petciones de escritura en el disco. (\texttt{r\_await} en \texttt{IOStat})
\end{itemize}

\subsection{Parámetros}
Los parámetros del sistema que influyen en las prestaciones serán:
\begin{itemize}
 \item Tipo de disco duro (SSD, magnético).
 \item Tipo de conexión del disco duro (IDE, SATA, USB 2.0, USB 3.0, \ldots).
 \item Tamaño de memoria caché del disco duro.
 \item Número de revoluciones por minuto del disco duro (en caso de ser magnético, ya que los discos de estado sólido no tienen
 velocidad de rotación - 3600, 5400, 7200).
 \item Tamaño de la partición que se empleará en el disco duro.
 \item Tipo de partición (NTFS, FAT32, \ldots).
 
 \cite{hddparam}
\end{itemize}

Los parámetros de la carga que influyen en las prestaciones son:
\begin{itemize}
 \item Tamaño de transferencia.
 \item Número de transferencias simultáneas.
\end{itemize}

\subsection{Factores}
Los factores seleccionados para este estudio son:

\begin{itemize}
 \item Tipo de disco duro.
 \item Tipo de conexión del disco duro.
 \item Tamaño de caché de disco duro.
 \item Revoluciones por minuto del disco duro.
 \item Número de transferencias simultáneas $n=1,2,\ldots5$
\end{itemize}

 A pesar de que se han seleccionado como factores los tipos de disco duro y conexión, y la velocidad de rotación, debido a 
 limitaciones de disponibilidad de hardware habrá casuísticas que no podrán testearse (SSD, discos de 7200 revoluciones). Los
 experimentos se realizarán con el equipo sobre el que se testeará liberado de cargas innecesarias, por lo que consideraremos
 el error introducido entre medidas en distintos discos duros y el error introducido entre pruebas al mismo disco duro con distinto
 número de transferencias simultáneas insignificante.
 
 \subsection{Técnica de evaluación}
 Se dispondrá de un sistema real con las siguientes características hardware y software para realizar el experimento:
 \begin{itemize}
  \item 
 \end{itemize}
 Por tanto la técnica de evaluación seleccionada es \textbf{medición sobre un sistema real}. Emplearemos el programa
 IOStat para efectuar la monitorización de la transferencia.

 \subsection{Carga de trabajo}
 La carga consistirá en efectuar copias simultáneas de varios ficheros al disco duro. Se ha optado por mantener
 un fichero de tamaño fijo (4,3 GB) que se copiará 1,2,3\ldots 5 veces de forma simultánea desde uno de los discos
 del sistema (el que contiene el SO) al disco duro a evaluar.%referenciar
 
 \subsection{Diseño experimental}
 Se empleará un diseño multi-factorial fraccionado, dado que no es posible evaluar todos los niveles de todos los factores
 que afectan al rendimiento.
 
 \subsection{Análisis de datos}
 
 \subsection{Presentación de datos}
 
 
 \section{Desarrollo experimental}




Dispondremos de varios discos duros:
\begin{itemize}
 \item %Completar
 \item %Completar
\end{itemize}





\newpage

\newpage
\printbibliography
\end{document}